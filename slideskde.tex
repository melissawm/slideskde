\documentclass[hyperref={pdfpagelabels=false}]{beamer}
\mode<presentation>
{
  \usetheme{Boadilla} % você pode escolher o tema de sua preferência! :)
  \setbeamercovered{transparent} 
}
\usepackage[portuguese]{babel} % usa palavras-chave em português
\usepackage[utf8]{inputenc} % permite incluir acentos no código

\title[KDE]{Como contribuir para o KDE sem escrever \\nenhuma linha de código}
%\author{}
\date{FISL14}

% Figura no slide-título. A posição desta figura no slide é determinada
% pelo tema escolhido em \usetheme.
\titlegraphic{\includegraphics[height=4cm]{konqi.png}}

\newif\ifplacelogo % criamos um condicional para usar o logotipo
                   % ou não, dependendo do slide
\placelogotrue 
\logo{\ifplacelogo\includegraphics[height=1.5cm]{konqi.png}\fi} % replace with your own command

\begin{document}
\placelogofalse
\begin{frame}
   % Title
   \titlepage
\end{frame}
\placelogotrue
\begin{frame}
   \frametitle{A comunidade KDE}
   \begin{itemize}
      \item 14 de outubro - 16 anos de KDE
      \item Estamos na versão 4.9.x
      \item Temos versões para desktop e mobile (plasma desktop, plasma netbook, plasma active)
      \item Cerca de 600 contribuidores (código) ativos atualmente
      \item Cerca de 2 mil contribuidores considerando todas as áreas (tradução, promoção, documentação etc)
      \item 202 listas de discussão
      \item 108 times de tradução, 30-40 idiomas estão razoavelmente bem traduzidos
      \item \emph{E não chegamos até aqui somente com a ajuda de programadores!}
   \end{itemize}
\end{frame}
\begin{frame}
   \frametitle{Por que contribuir com o KDE?}
   \begin{itemize}
      \item Questões de sociabilidade: Para se envolver mais diretamente na comunidade que produz o software que você usa e gosta
      \item Questões de conhecimento: Para aprender/ensinar coisas novas seja qual for a sua área de atuação; desenvolver suas habilidades
      \item Questões técnicas: Para ajudar a produzir softwares livres cada vez melhores.
   \end{itemize}
\end{frame}
\begin{frame}
   \frametitle{Como contribuir?}
   \begin{itemize}
      \item Artwork/Design
      \item Promoção
      \item Tradução
      \item Documentação
      \item Teste
      \item Acessibilidade
   \end{itemize}
\end{frame}
\begin{frame}
   \frametitle{Artwork/Design}
   \begin{center}
      \begin{block}{}
         \begin{center}
            O KDE também precisa agradar aos olhos!
         \end{center}
      \end{block}
   \end{center}
   \begin{itemize}
      \item Icones
      \item Telas de login
      \item Wallpapers
      \item Temas
      \item Artes de camisetas/Eventos/Brindes
   \end{itemize}
\end{frame}
\begin{frame}
   \frametitle{Artwork/Design}
   Por onde começar?
   \begin{itemize}
      \item Se informar sobre as diretrizes de arte/identidade visual do KDE
      \item Entrar no canal {\tt{\#kde-artists}} no {\tt{irc.freenode.net}}
      \item Lista de discussão
   \end{itemize}
\end{frame}
\begin{frame}
   \frametitle{Promoção}
   \begin{center}
      \begin{block}{}
         \begin{center}
            O KDE também precisa ser divulgado!
         \end{center}
      \end{block}
   \end{center}
   \begin{itemize}
      \item Escrever textos sobre o KDE em blogs e sites e/ou nos sites do próprio KDE
      \item Participar de eventos representando o KDE
      \item Ajudar a encontrar novos colaboradores
      \item Divulgar o KDE nas redes sociais, etc
   \end{itemize}
\end{frame}
\begin{frame}
   \frametitle{Promoção}
   Por onde começar?
   \begin{itemize}
      \item Participar da lista de discussão do promo
      \item Entrar no canal {\tt{\#kde-promo}} no {\tt{irc.freenode.net}}
   \end{itemize}
\end{frame}
\begin{frame}
   \frametitle{Tradução}
   \begin{center}
      \begin{block}{}
         \begin{center}
            O KDE também precisa ser traduzido para nossa língua!
         \end{center}
      \end{block}
   \end{center}
   \begin{itemize}
      \item Tradução offline de GUI e DOC (Lokalize)
      \item Tradução online e offline de textos e wikis do KDE.
   \end{itemize}
\end{frame}
\begin{frame}
   \frametitle{Tradução}
   Por onde começar?
   \begin{itemize}
      \item Participar da lista de discussão
      \item Entrar no canal {\tt{\#kde-i18n}} no {\tt{irc.freenode.net}}
   \end{itemize}
\end{frame}
\begin{frame}
   \frametitle{Documentação}
   \begin{center}
      \begin{block}{}
         \begin{center}
            O KDE também precisa que seus programas sejam melhor entendidos!
         \end{center}
      \end{block}
   \end{center}
   \begin{itemize}
      \item Ajuda de contexto
      \item Manual dos aplicativos
      \item Documentação de API
      \item Dicas 
   \end{itemize}
\end{frame}
\begin{frame}
   \frametitle{Documentação}
   Por onde começar?
   \begin{itemize}
      \item Participar da lista de discussão
      \item Entrar no canal {\tt{\#kde-docs}} no {\tt{irc.freenode.net}}
   \end{itemize}
\end{frame}
\begin{frame}
   \frametitle{Testes}
   \begin{center}
      \begin{block}{}
         \begin{center}
            O KDE também precisa de caçadores de bugs!
         \end{center}
      \end{block}
   \end{center}
   \begin{itemize}
      \item Verificar bugs e reportá-los
   \end{itemize}
   
   Por onde começar?
   \begin{itemize}
      \item Visitar o site: {\tt{https://bugs.kde.org/}}
      \item Entrar no canal {\tt{\#kde-bugs}} no {\tt{irc.freenode.org}}
   \end{itemize}
\end{frame}
\begin{frame}
   \frametitle{Acessibilidade}
   \begin{center}
      \begin{block}{}
         \begin{center}
            O KDE também precisa ser acessível a todos!            
         \end{center}
      \end{block}
   \end{center}
   \begin{itemize}
      \item Avaliação e documentação dos recursos de acessibilidade do KDE.
   \end{itemize}

   Por onde começar?
   \begin{itemize}
      \item Participar da lista de discussão
      \item Entrar no canal {\tt{\#kde-accessibility}} no {\tt{irc.freenode.net}}
   \end{itemize}
\end{frame}
\begin{frame}
   \frametitle{Join the game!}
   \begin{center}
      \begin{block}{}
         \begin{center}
            {\tt{http://jointhegame.kde.org}}
         \end{center}
      \end{block}
   \end{center}
\end{frame}
\begin{frame}
   \frametitle{Links Importantes}
   \begin{itemize}
      \item Listas de discussão: {\tt{https://mail.kde.org/mailman/listinfo/kde-br}}
      \item Canais do irc: {\tt{\#kde}}, {\tt{\#kde-brasil}}
      \item Redes sociais: {\tt{{@}kdebrasil}}
      \item {\tt{techbase.kde.org}} e {\tt{userbase.kde.org}}
   \end{itemize}
\end{frame}
\begin{frame}
   \frametitle{}
   \begin{center}
      \begin{block}{}
         \begin{center}
            \Huge{Perguntas?}
         \end{center}
      \end{block}
   \end{center}
\end{frame}
\end{document}